\begin{DoxyNote}{Note}
I can include an image anywhere I want in my Doxygen comments by simply using the \char`\"{}\textbackslash{}image\char`\"{} command 
\end{DoxyNote}
\hypertarget{index_intro}{}\section{Program Introduction}\label{index_intro}
The {\bfseries{Shapes Project}} is an educational application designed to illustrate object-\/oriented programming concepts, particularly focusing on inheritance and polymorphism. It provides a simple framework for managing geometric shapes such as circles and squares.

The project consists of three main classes\+:
\begin{DoxyItemize}
\item {\bfseries{\mbox{\hyperlink{class_shape}{Shape}}}}\+: The base class representing a generic geometric shape. It provides functionality for setting and retrieving the name and color of a shape, as well as calculating its perimeter, area, and overall dimension.
\item {\bfseries{\mbox{\hyperlink{class_circle}{Circle}}}}\+: A derived class inheriting from the \mbox{\hyperlink{class_shape}{Shape}} class, representing a circle shape. It includes methods for setting and retrieving the radius of the circle, as well as calculating its perimeter and area.
\item {\bfseries{\mbox{\hyperlink{class_square}{Square}}}}\+: Another derived class from \mbox{\hyperlink{class_shape}{Shape}}, representing a square shape. Similar to \mbox{\hyperlink{class_circle}{Circle}}, it provides methods for setting and retrieving the side length of the square, along with calculating its perimeter and area.
\end{DoxyItemize}

Through this project, students can learn how to design and implement class hierarchies using inheritance, utilize polymorphism to work with objects of different types through a common interface, and understand the practical application of object-\/oriented principles in real-\/world scenarios.

Major Doxygen features demonstrated in this project include\+:


\begin{DoxyItemize}
\item Project/\+Application description and documentation on the main page
\item Commenting classes and methods using Doxygen
\item Ignoring source code comments in C or C++ style while constructing project documentation
\end{DoxyItemize}

External links can also be included in Doxygen comments, such as\+:
\begin{DoxyItemize}
\item \href{http://www.example.com/}{\texttt{ Example Website}}
\end{DoxyItemize}



 \hypertarget{index_notes}{}\section{Special Release Notes}\label{index_notes}
This section can be used to provide release notes for each version of the software, including added features, fixed bugs, and other changes.



 \hypertarget{index_requirements}{}\section{Project Requirements}\label{index_requirements}
The project requirements can be included directly in the Doxygen documentation, either as text or by including the contents of a requirements document.



 \begin{DoxyRefDesc}{Todo}
\item[\mbox{\hyperlink{todo__todo000001}{Todo}}]\mbox{[}optionally include text about more work to be done\mbox{]} 

Give each todo item its own line 

Assigment 7\end{DoxyRefDesc}




 \begin{DoxyRefDesc}{Bug}
\item[\mbox{\hyperlink{bug__bug000001}{Bug}}]\mbox{[}optionally include known bugs and limitations within the project here\mbox{]}
\begin{DoxyItemize}
\item B\+UG \+: This is known bug \#1
\item B\+UG \+: This is known bug \#2
\item I\+S\+S\+UE \+: Here is a known issue/limitation with the project
\end{DoxyItemize}\end{DoxyRefDesc}




 \hypertarget{index_funfacts}{}\section{Fun Facts and Features}\label{index_funfacts}

\begin{DoxyItemize}
\item The project includes dynamic memory allocation for managing objects of \mbox{\hyperlink{class_circle}{Circle}} and \mbox{\hyperlink{class_square}{Square}} classes.
\item The \mbox{\hyperlink{class_shape}{Shape}} class serves as the foundation for polymorphic behavior across different geometric shapes.
\item Doxygen comments provide comprehensive documentation for easy understanding and maintenance of the codebase.
\item Project was done alone as a group member left -\/Project used G\+IT \& G\+I\+T\+U\+HB
\end{DoxyItemize}



 \hypertarget{index_version}{}\section{Current Version of the Shapes Project\+:}\label{index_version}

\begin{DoxyItemize}
\item \begin{DoxyAuthor}{Author}
Muhammad Elsoukkary 
\end{DoxyAuthor}

\item \textbackslash{}student\# 8826383 
\item \begin{DoxyVersion}{Version}
1.\+00.\+00 
\end{DoxyVersion}

\item \begin{DoxyDate}{Date}
2024-\/03-\/17 
\end{DoxyDate}

\item \begin{DoxyWarning}{Warning}
Improper use of the Shapes Project may result in unexpected behavior. 
\end{DoxyWarning}

\end{DoxyItemize}